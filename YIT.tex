\documentclass{article}

\usepackage[no-math]{fontspec}

\usepackage{tikz}
\usepackage{todonotes}
\def\faire{\todo[inline]{TODO!}}

\newcommand{\tran}[3]{%
	\noindent\textsf{#1:} (#2). #3%
	\medskip\newline
}
% easy as pie:
% #1: name of the language
% #2: Yoneda lemma in the language of #1
% #3: statement of the lemma

\setmainfont{Quivira}

\begin{document}
\begin{center}
Yoneda lemma \\
in every language
\end{center}
\section{Introduction}
\faire
\section{Yoneda lemma}
\faire

% \tran{Simple English}{Yoneda lemma}{\dots}
\tran{{\L}engua vèneta (ISO 639-3 vec)}{El lema de Yoneda}{
	Toì na categoria pìco{\l}a $\mathcal C$ e un fontor $F$ de sta categoria `nte la categoria dei insièmi. Al\'ora, comunque che se toga n'ogeto $X$ de $\mathcal C$ gh'è n'isomorfismo (naturae ent'el sò argomento) tra l'insième dee trasforma{\ss}ioni naturai $\hom(-,X)\to F$ e l'insième $FX$, fisà de la rego{\l}a 
	\[
	\Big(\xi : \hom(-,X)\Rightarrow F\Big) \mapsto \xi_X(1_X)	
	\]
	(sta fonsion ea xe bijetiva).
}
\tran{Sicilianu (ISO 639-3 scn)}{Lemma ri Yuneda}{
	Aviss'a pigghiari na categuria $\mathcal C$, e un funturi $F$ ri sta categuria rint'agl'insèmi. Pi tutti l'oggetti $X$ ri $\mathcal C$, avimu na biggezione naturale 'nta l'insèmi ri tutte le trasformazioni naturali $\hom(-,X)\to F$ e l'insèmi $FX$, fissatu ri la reggola
	\[
		\Big(\xi : \hom(-,X)\Rightarrow F\Big) \mapsto \xi_X(1_X)	
	\]
	(ssa funzioni iè biggettiva).
}
\tran{Esperanto (ISO 639-3 epo)}{Lemo el Yoneda}{
	Por \^{c}iuj kategorio $\mathcal C$ kaj functo $F$ de la kategorio $\mathcal C$ en la kategorio de aroj, kaj por \^{c}iuj objektoj $X$ el $\mathcal C$ estas reciproke unuvalora sur\^{j}eto inter la aro de naturaj transformoj $\hom(-,X)\to F$ kaj la aro $FX$, specifita de funkcio
	\[
		\Big(\xi : \hom(-,X)\Rightarrow F\Big) \mapsto \xi_X(1_X).
	\]
}
\tran{Zenéise (ISO 639-3 lij)}{Lémma de Yoneda}{
    Segge $\mathcal C$ una categuia picenina e $F$ ün funtu' da sta categuia in ta' categuia di insiemmi. Al\^{o}a pe tutte e cose $X$ in $\mathcal C$ gh'è üna biiessiun naturale tra l'insiemme de trasfurmasiun naturali $\hom(-,X)\to F$ e l'insiemme $FX$, fisa da-a regula
    \[
		\Big(\xi : \hom(-,X)\Rightarrow F\Big) \mapsto \xi_X(1_X).
	\]
    (sta fonçiún a l'è biiettiva).
}
\tran{Napulitane (ISO 639-3 nap)}{Lemma e' Yoneda}{
	Piʝətə $\mathcal C$ 'na categuriə piccerella e $F$ 'nu funtorə partenn a' chesta categuriə inte agl'insiemə. Allor pe' tutti quanti l'oggetti $X$ e' $\mathcal C$ ce' sta 'na funzionə ca po' turnà arrete partenn a' l'insiemə de' trashformazionə naturalə $\hom(-,X)\to F$ e l'insiemə $FX$ fissat da' regula
	\[
		\Big(\xi : \hom(-,X)\Rightarrow F\Big) \mapsto \xi_X(1_X).
	\]
  (chesta funzionə po' turnà arrete).
}
\end{document}